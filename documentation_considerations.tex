\documentclass{article}
\begin{document}
\title{Documentation considerations}

I have studied the use of TeX on documentation.



Why I consider TeX, rather than Markdown, HTML or RST has
following reasoning:

\begin{itemize}
\item
    TeX syntax is older than Markdown or reStructuredText.
        It was designed for more generic purpose of
        typesetting documents. This better matches what I am
        supposed to do with the language.
\item
    TeX is more generic than Markdown or reStructuredText,
        with less of structures meant for constructing
        certain kind of texts.
\item
    While I've been using Markdown in my blog posts, it has
        shown that it is not entirely good for constructing
        well-layouted and structured HTML pages. It
        separates the layout and content too well, even if
        by design guidelines the layout and style SHOULD
        FOLLOW the content.
\item
    reStructuredText seem to share many problems of the
        Markdown by trying to be too human readable as plain
        text documentation.
\item
    HTML has TeX-like qualities, but it is ugly syntax and
        would require further processing to allow targeting
        for multiple purposes.
\end{itemize}

If you create a subset of TeX that has structures that can
be understood by all your document processing targets, it
could be very versatile platform for writing documentation.

For example, there already exist many important pieces for
writing hypertext documents, such as:

\begin{verbatim}
\href{http://domain}{description}
\label{location}
\ref{location}
\end{verbatim}

TeX doesn't come without problems though.

TeX is meant to be read by a document processor that reads
it command-by-command in from a text stream, considering
every character as a command, that causes it to be perceived
as turing complete. The document processor can change the
meaning of its input stream. This is considered notorious
because TeX can be only parsed by TeX.

I consider it much worse that TeX has up to 10 control
characters, with many of them being ordinary elements when
typing modern programming related texts.

\verb= # % & / ^ _ { } ~ $ =

You can stuff most of code samples into verbatim blocks and
prevent many issues here. But things like underscore can
provide to be quite problematic when typing out variable
names.

But then, you could add a macro for constructing variable
names and avoid these problems. That macro would also work
for linking those names into reference.

The backslash -character is easy to type on american
keyboards but difficult on european keyboards. This would
propose to change the symbol.

Another gripe is that TeX does not consider the layout of
the text. This would be very good quality for readability.

Here's how I think a Lever documentation could look like:

\begin{verbatim}
#section
    Title text here, goes to the end of a block.

Here you see that title gobbles an indented block. The
indentation must increase on next line.

The indentation wouldn't be used for potentially long blocks
of text, because that is rather ugly and demeaning to
process on an editor when it could go on for pages far.

The main differences to TeX:

#begin itemize
#item
Different macro -letter. Differentiating from TeX
#item
3 special characters: #hash #lbra #rbra
#item
Braces still used for grouping at places.
#item
Indentation an available structuring element for
simple elements such as titles. Optional elsewhere.
#item
Language extended before processing, not during
processing.
#item 
Hyperref elements very much like in TeX:
#href{http://example.org}{Test URL}
#item
Forms clear distinction between TeX.
#item
Similar to write as TeX on british keyboards. On european
and american keyboards holds similar difficulties.
#item
Unformatted texts are less likely to trigger actions from
the special characters list.
#item
High-level enough description to support all the
targets we need:
    #begin itemize
    #item HTML
    #item TeX & PDF documents
    #item References in the runtime
    #end
#end
\end{verbatim}

\end{document}
